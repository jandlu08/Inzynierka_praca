\chapter{Wstęp}
\section{Cel i zakres pracy}
Celem pracy jest zaprojektowanie i zaimplementowanie aplikacji internetowej służącej do pomagania w układaniu diety na podstawie dziennego zapotrzebowania kalorycznego użytkownika. Powodem powstania tej pracy jest niska wiedza i świadomość ludzi w dziedzinie zdrowego oraz zbilansowanego żywienia.\cite{zywienie} Poza tym wpływ na wybór technologii miały zainteresowania autora, chęć podniesienia własnych umiejętności oraz obecne trendy.\cite{stack} Aplikacja ma działać na każdej współczesnej przeglądarce internetowej. Backend został wykonany na platformie ASP.NET Core 3.0, frontend w Angular 8.0, natomiast jako baza danych wykorzystywało  PostgreSQL 11.4. API oparte zostało na otwartym standardzie GraphQL. W zakres pracy wchodzi:
\begin{itemize}
    \item zapoznanie się z potrzebnymi bibliotekami i platformami programistycznymi,
    \item analiza potrzeb funkcjonalnych i niefunkcjonalnych użytkowników,
    \item projekt i konfiguracja bazy danych,
    \item projekt aplikacji,
    \item instrukcja użytkowania,
    \item krótka analiza wydajności aplikacji.
  %  \item testowanie aplikacji.
\end{itemize}
\section{Układ pracy}
\todo{Na końcu zrobić}


